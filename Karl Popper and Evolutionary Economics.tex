\documentclass{sintefbeamer}

% packages, font, color, and newcommands
\usepackage{amsfonts, amsmath, oldgerm, lmodern}
\usefonttheme{serif}
\usepackage{adjustbox}
\usepackage{ragged2e}
% meta-data
\title{Karl Popper and Evolutionary Economics}
\subtitle{A review}
\author{\href{mailto:daniela.cialfi@unich.it}{Daniela Cialfi}}
\date{July 4th, 2022}
\titlebackground{images/background}

% document body
\begin{document}

\maketitle

\begin{frame}
\justifying{This presentation has the aim to analyse the Nelson and Winter’s economics evolutionary theory (\cite{Nelson1}) under the Austrian philosopher of science Karl Popper's methodology and epistemology framework.}\\


\begin{columns}
\begin{column}{0.7\textwidth}
\justifying{So, the answer to this dilemma discovers that the philosophical fundamentals of Nelson and Winter theory. This is what, as we will see throughout this presentation, Popper called the \textit{method of situational logic} or \textit{situational analysis}.}
\end{column}
\begin{column}{0.3\textwidth}
\includegraphics[scale=0.15]{images/Karl_Popper.jpg}\\
\end{column}
\end{columns}




%\includegraphics{images/Karl_Popper}

\end{frame}

\section{Introduction}

\begin{frame}{The origin}{\thesection \, \secname}

\justifying{1959 could be considered as a important date for both Philosophy of science and Economics because we have}\\

\begin{columns}
\begin{column}{0.7\textwidth}
\justifying{\textbf{Karl Popper's view on the origin and evolution of life}
\begin{column}{0.3\textwidth}
\justifying{\textbf{Debate related to Solow investigation about the economic growth issue}}
\end{column}
\end{columns}
\end{frame}

\begin{frame}{Beamer vs. PowerPoint}
Compared to PowerPoint, using \LaTeX\ is better because:
\begin{itemize}
\item It is not What-You-See-Is-What-You-Get, but
What-You-\emph{Mean}-Is-What-You-Get:\\
you write the content, the computer does the typesetting
\item Produces a \texttt{pdf}: no problems with fonts, formulas,
      program versions
\item Easier to keep consistent style, fonts, highlighting, etc.
\item Math typesetting in \TeX\ is the best:
\begin{equation*}
\mathrm{i}\,\hslash\frac{\partial}{\partial t} \Psi(\mathbf{r},t) =
-\frac{\hslash^2}{2\,m}\nabla^2\Psi(\mathbf{r},t)
+ V(\mathbf{r})\Psi(\mathbf{r},t)
\end{equation*}

\end{itemize}
\end{frame}

\section{Karl Popper on the evolutionary theory}

\begin{frame}[fragile]{Selecting the Class}
After the last update to the graphic profile, the \texttt{sintef} theme for
Beamer has been updated into a full-fledged class.
To start working with \texttt{sintefbeamer}, start a \LaTeX\ document with the
preamble:
\begin{block}{Minimum SINTEF Beamer Document}
\begin{lstlisting}[language=TeX]
\documentclass{sintefbeamer}
\begin{document}
\begin{frame}{Hello, world!}
\end {frame}
\end{document}
\end{lstlisting}
\end{block}
\end{frame}

\begin{frame}[fragile]{Title page}
To set a typical title page, you call some commands in the preamble:
\begin{block}{The Commands for the Title Page}
\begin{lstlisting}[language=TeX]
\title{Sample Title}
\subtitle{Sample subtitle}
\author{First Author, Second Author}
\date{Defaults to today's}
\end{lstlisting}
\end{block}
You can then write out the title page with \verb|\maketitle|.

You can set a different background image than the default one with the
\verb|\titlebackground| command, set before \verb|\maketitle|.

In the \texttt{backgrounds} folder, you can find a lot of standard backgrounds
for SINTEF presentation title pages.

\end{frame}

\begin{frame}[fragile]{Writing a Simple Slide}
\framesubtitle{It's really easy!}
\begin{itemize}[<+->]
\item A typical slide has bulleted lists
\item These can be uncovered in sequence
\end{itemize}
\begin{block}{Code for a Page with an Itemised List}<+->
\begin{lstlisting}[language=TeX]
\begin{frame}
  \frametitle{Writing a Simple Slide}
  \framesubtitle{It's really easy!}
  \begin{itemize}[<+->]
    \item A typical slide has bulleted lists
    \item These can be uncovered in sequence
  \end{itemize}
\end{frame}\end{lstlisting}
\end{block}
\end{frame}

\begin{frame}[fragile]{Adding images}
\begin{columns}
\begin{column}{0.7\textwidth}
Adding images works like in normal \LaTeX:
\begin{block}{Code for Adding Images}
\begin{lstlisting}[language=TeX]
\usepackage{graphicx}
% ...
\includegraphics
[width=\textwidth]{images/default}
\end{lstlisting}
\end{block}
\end{column}
\begin{column}{0.3\textwidth}
\includegraphics
[width=\textwidth]{images/default}\\
\end{column}
\end{columns}
\end{frame}

\begin{frame}[fragile]{Splitting in Columns}
Splitting the page is easy and common;
typically, one side has a picture and the other text:
\begin{columns}
\begin{column}{0.6\textwidth}
This is the first column
\end{column}
\begin{column}{0.3\textwidth}
And this the second
\end{column}
\end{columns}
\begin{block}{Column Code}
\begin{lstlisting}[language=TeX]
\begin{columns}
    \begin{column}{0.6\textwidth}
        This is the first column
    \end{column}
    \begin{column}{0.3\textwidth}
        And this the second
    \end{column}
    % There could be more!
\end{columns}
\end{lstlisting}
\end{block}
\end{frame}

\begin{frame}[fragile]
\frametitle{Fonts}
\begin{itemize}
\item The paramount task of fonts is being readable
\item There are good ones...
  \begin{itemize}
  \item {\textrm{Use serif fonts only with high-definition projectors}}
  \item {\textsf{Use sans-serif fonts otherwise (or if you simply prefer them)}}
  \end{itemize}
\item ... and not so good ones:
  \begin{itemize}
  \item {\texttt{Never use monospace for normal text}}
  \item {\frakfamily Gothic, calligraphic or weird fonts: should always: be
  avoided}
\end{itemize}
\end{itemize}
\end{frame}

\begin{frame}[fragile]{Look}
\begin{itemize}
\item To change the colour of the title dash, give one of the class options
      \texttt{cyandash} (default), \texttt{greendash}, \texttt{magentadash},
      \texttt{yellowdash}, or \texttt{nodash}.
\item To change between the light and dark themes, give the class options
      \texttt{light} (default) or \texttt{dark}. It is not possible to switch
      theme for one slide because of the design of Beamer---and it's probably a
      good thing.
\item To insert a final slide, use \verb|\backmatter|.
\item The aspect ratio defaults to 16:9, but you can change it to 4:3 for old
      projectors by passing the class option \texttt{aspectratio=43}; any other
      values accepted by Beamer are also possible.
\end{itemize}
\end{frame}

\section{Karl Popper on the evolutionary theory}
\section{Karl Popper’s approach to Evolutionary economics}
\section{Summary}
\begin{frame}{References}

\end{frame}
\begin{frame}
\frametitle{Good Luck!}
\begin{itemize}
\item Enough for an introduction! You should know enough by now
\item If you have corrections or suggestions,
\hrefcol{mailto:federico.zenith@sintef.no}{send them to me!}
\end{itemize}
\end{frame}

\begin{frame}[allowframebreaks]{References}

  \bibliography{demo}
  \bibliographystyle{abbrv}

\end{frame}
\backmatter

\end{document}
